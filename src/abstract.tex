The exhaust of particles and heat in the boundary of contemporary magnetic confinement experiments remains to this day a major obstacle on the road to
commercially viable fusion energy production. It is recognized, that coherent structures of hot and dense plasma, called blobs or filaments, are the dominant mechanism for cross-field particle transport. These filaments are created by plasma turbulence at the outboard midplane and move radially outwards driven by interchange motions. This leads to high average particle densities and relative fluctuation levels in the scrape-off layer, which increases plasma-wall interactions.

Time series of the plasma density measured at a fixed point using either Langmuir probes or gas puff imaging have shown highly intermittent fluctuations across a variety of devices, plasma parameters and confinement modes. Recent statistical analysis of measurement data time series has revealed that the fluctuations are well described as a superposition of uncorrelated exponential pulses with fixed duration and exponentially distributed pulse amplitudes, arriving according to a Poisson process.  

Due to the complexity of the
physics involved in the boundary of fusion devices, numerical simulations are utilized to gain an accurate description of scrape-off layer plasmas. This approach requires a validation metric for simulations of plasma turbulence such as the statistical framework based on filtered Poisson processes. In this thesis, well-established models for scrape-off layer plasmas are analyzed. These models use two-fluid equations simulating plasma evolution in the two-dimensional plane perpendicular to the magnetic field. Time series of the plasma density are measured at a fixed point and their fluctuation statistics are compared to experimental measurements utilizing the statistical framework. This includes probability density functions, power spectral densities and conditionally averaged waveforms. In addition, simulations of a population of seeded blobs are performed in order to study the effects of blob interactions. It is shown that the fluctuation statistics of single-point measurements in simple numerical models stand in excellent agreement with their experimental counterparts. This work thereby sets a new standard and methodology for validating scrape-off layer turbulence simulations. 





\begin{comment}
Statistical properties of numerical simulations of scrape-off layer (SOL) plasmas are investigated. Single point measurements of the plasma density and temperature of two established fluid models are studied and compared to the predictions of the Filtered Poisson Process (FPP), a phenomenological model describing all major statistical properties of experimental measurements of fluctuations in SOL plasmas.

We find that the idealized interchange model exhibits the statistical properties of SOL
fluctuations and the FPP model only to some extend. The probability density functions (PDF) for the temperature and
radial velocity fluctuations change from a normal distribution in the center
of the simulation domain to a distribution with an exponential tail at higher
radial positions. The power spectral densities (PSD) have an exponential shape which can be attributed to underlying Lorentzian pulses. The time
series of the temperature show periods of strongly intermittent fluctuations
with large bursts, interrupted by quiescent periods of quasi-periodic fluctuations.
This effect can be attributed to the generation of a sheared mean flow through the fluid layer resulting in predator-prey-like dynamics of
the energy integrals.

Analytical work on a shot noise process
with periodic arrivals reveals that the PSD
of this process with Lorentzian pulses and exponentially distributed amplitudes has an exponential shape with a Dirac comb of decaying amplitudes. It is shown that that moderate deviations from perfect periodicity destroys the Dirac comb as it
leads to a broadening of its peaks and the decrease of the peak amplitudes
for higher harmonics, resembling the PSD of the idealized interchange model.

Equivalent analysis on time series of the reduced Braginskii fluid model show that the PDFs of stationary time series of the plasma density have an
exponential tail for high radial positions. The average burst or pulse shape is well described by a
two-sided exponential function. The PSD of the particle density is that of the
average pulse shape and does not change with radial position. The amplitudes
and the waiting times between two consecutive arrivals are exponentially distributed. The results of this model thereby stand in perfect agreement with the predictions of
the FPP model. The
presented statistical framework defines a new validation metric for boundary turbulence simulations.

Lastly, the interaction of filaments in the SOL is investigated in the reduced Braginskii model utilizing a blob tracking algorithm. We observe that a high level of filament interaction  leads to an increase in the average radial velocity, as filaments interact with the electrostatic potential of one another. For all investigated levels of filament interaction, the size-velocity relationship of these structures follow established scaling laws. The relevance of isolated filament simulations for complex turbulence models is thereby displayed.\textcolor{red}{Maybe reduce amount of information since abstract became very long!}
\end{comment}